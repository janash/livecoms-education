
\textbf{Contributed by Jessica A. Nash}

 The Molecular Sciences Software Institute (MolSSI) serves and enhances software development efforts in the computational molecular sciences (CMS). MolSSI Education is the institute's division that trains students and researchers in computer programming and software development practices. Our three main focus areas are (1) Programming and Software Development; (2) High-performance computing (HPC); and (3) Faculty and curriculum development \cite{NashCISE}. 

MolSSI Education’s educational material and workshops target a range of CMS students and researchers including undergraduates, graduate students, faculty, and professionals in CMS. 
Our online lessons are conveniently accessible through user-friendly Jupyter notebooks and web pages, which provide step-by-step examples of lesson content. 
Tutorials and workshops are hands-on, focusing on giving the student practical experience and skills. 
Online materials include our Python Data and Scripting for Computational Molecular Science, Best Practices in Software Development, Fundamentals of Parallel Programming, and many others (see \href{https://education.molssi.org/resources}{https://education.molssi.org/resources} for a complete list). 

We strive to collaborate with other institutes, communities, and resources and regularly hold workshops with academic research groups and universities. The MolSSI has also hosted academic faculty on sabbatical, resulting in the development of specialized learning materials for CMS researchers \cite{Craig2022}.  

All MolSSI Education lessons are developed with open licensing and are hosted on GitHub through the \href{https://github.com/molssi-education}{MolSSI Education GitHub Organization}. We welcome external collaboration and contributions through GitHub Pull Requests. If you would like to discuss workshop opportunities with MolSSI, you can contact the MolSSI Education Lead, Dr. Jessica Nash (janash@vt.edu), or MolSSI's Co-Director for Education, Training, and Faculty Development, Prof. Ashley McDonald (armcdona@calpoly.edu)