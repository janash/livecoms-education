\textbf{Contributed by Dominique Sydow and Andrea Volkamer}

TeachOpenCADD is a free online platform launched in 2019 that presents common topics from computer-aided drug design (CADD) and how to solve them programmatically in Python with open-source data and packages \cite{TeachOpenCADD2019, TeachOpenCADD2022, TeachOpenCADDKinaseEdition}. The current lessons range from cheminformatics over structural bioinformatics to molecular modeling. The material also exemplifies how to query domain-relevant databases such as ChEMBL \cite{Mendez_2018_ChEMBL}, PubChem \cite{Kim_2021_PubChem}, and the PDB \cite{Burley_2020_PDB}. 
Each lesson is presented in the form of an interactive Jupyter notebook \cite{Kluyver_2016_Jupyter}, which combines detailed theoretical background and executable code with a discussion and a quiz. The material is suitable for students and teachers from different disciplines and programming levels ranging from beginners to advanced users. It can be applied to self-study and classroom teaching but also serves as a basis for debarking on research questions \cite{TeachOpenCADDTeaching}. 

The TeachOpenCADD material is available in many forms: (i) The TeachOpenCADD website \cite{TOC_Website} allows for instant browsing of the full content in read-only mode. (ii) Users can execute and modify the material either online via the Binder \cite{binder} support or locally via the TeachOpenCADD conda package \cite{conda_forge_community_2015_4774216, conda_forge_toc} as described in detail in the TeachOpenCADD documentation \cite{TOC_Documentation}. (iii) If users are looking for a coding-free solution, the cheminformatics-focused topics are available as GUI-based KNIME workflows \cite{TeachOpenCADDKNIME2019}. The TeachOpenCADD platform is actively maintained and continuously extended with new content (e.g. the kinase edition in 2022 \cite{TeachOpenCADDKinaseEdition} and a deep learning edition under current development). It is open to contributions, bug reports, and questions from the community via its GitHub repository \cite{TOC_Website}.