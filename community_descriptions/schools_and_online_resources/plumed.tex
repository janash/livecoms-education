\textbf{Contributed by Massimiliano Bonomi, Carlo Camilloni, Gareth A. Tribello, Giovanni Bussi}

The PLUMED consortium\cite{plumed2019promoting} is a community born around the PLUMED open source software (\href{https://www.plumed.org}{www.plumed.org})\cite{tribello2014plumed}, a library to perform enhanced sampling and free energy calculations, and to analyze the data produced by molecular dynamics (MD) simulations. PLUMED works together with several molecular simulation packages, including but not limited to GROMACS, NAMD, OpenMM, Amber, CP2K, and Quantum ESPRESSO. The PLUMED consortium  is an open community that includes current and past PLUMED developers, contributors, and all those researchers whose work builds in part on PLUMED and at the same time drives the development and dissemination of the software. The mission of the consortium is to transform the way researchers communicate the protocols that are used in their MD simulations, in order to maximize the impact of new research and promote the highest possible standards of scientific reproducibility. 

To achieve these goals, the PLUMED consortium coordinates two main initiatives. The first one is a public repository named PLUMED-NEST (\href{https://www.plumed-nest.org}{www.plumed-nest.org}) where contributors can share examples (including input and data files) to promote reproducibility and transparency in computational molecular science. The second initiative is a series of tutorials, called PLUMED Masterclasses, that teach molecular simulation techniques ranging from basic algorithms to advanced enhanced sampling methods. Recordings of lectures about these techniques can be found on the Youtube channel @plumedorg1402, and all the teaching material, including exercises, solutions, and slides, are available at www.plumed.org/masterclass. Further training and interactions between members of the PLUMED community revolve around the PLUMED forum (plumed-users@googlegroups.com) and the periodic, in-person schools and user meetings, which give the opportunity to new and experienced users to meet and discuss specific use cases.