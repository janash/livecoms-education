\textbf{Contributed by Patrick Norman}

The \href{https://doi.org/10.30746/978-91-988114-0-7}{eChem project} offers a platform for interactive studies in computational chemistry based on Jupyter notebooks and the use of modern Python-driven electronic structure softwares such as VeloxChem  and Gator.  
The notebooks are compiled into a Jupyter Book that targets a wide audience ranging from undergraduate students to experienced researchers.
The central educational idea of eChem is to leverage the Python-layer exposure of data structures in VeloxChem and other software modules to enable the study of quantum chemistry in a similar manner that NumPy (or Matlab) can be used in studies of linear algebra. Presentations of theory and equations are intertwined with notebook cells showing the form these equations take in Python code implementations and fundamental concepts such as two-particle densities and electron correlation can be made visual with use of Matplotlib illustrations. 

While the eChem book is well suited for self-directed learning, the material can also be used to put together workshops led by experienced instructors or as components in standard courses in the chemistry curriculum.  Students are recommended to use conda to install the software on their own laptops running Windows, MacOS, or Linux. 
The eChem book is published under the Creative Commons Attribution-ShareAlike license with the source code available on GitHub.  Community contributions to the project are welcome both as comments and corrections as well as new book sections. The former category is handled through GitHub whereas the latter is initiated by e-mail correspondence to one of the authors.