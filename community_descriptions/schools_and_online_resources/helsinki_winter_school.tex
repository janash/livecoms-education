\textbf{Contributed by Susi Lehtola}

The \href{http://www.chem.helsinki.fi/ws.html}{Helsinki Winter School in
Theoretical Chemistry} has been organized annually
since 1985 at the \href{http://www.chem.helsinki.fi/}{Department of Chemistry}
of the \href{www.helsinki.fi/}{University of Helsinki}, Finland, with the
exception of 2020 and 2021 due to the coronavirus pandemic. The theme of the
Winter School changes from year to year, in order to cover various fields of
theoretical chemistry as well as possible. For example, the last four schools
covered quantum computers for chemistry (2022), inorganic chemistry (2019),
machine learning in theoretical chemistry (2018), and current trends in
electronic structure methods (2017). The school aims to invite the leading 
experts in the field, guaranteeing lectures of high scientific quality. The
lecture materials are typically only made available to the workshop attendees,
so that lecturers feel more free to present unpublished results.

The winter school is typically organized in late October to early December as a
four-day event, and usually has around 80 participants, around half of whom come
from abroad. The school is targeted for PhD students and young researchers, but
they usually also include a few more established researchers and professors; the
presence of the more senior personnel often leads to lively scientific
discussions during the lectures. Participation in the winter school is free;
however, attendees need to pay their own travel and accomodation. The school
also features a poster session, which welcomes posters representing any area of
theoretical chemistry.