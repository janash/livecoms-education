\textbf{Contributed by Neeraj Rai}

The Institute of Computational Molecular Science Education (i-CoMSE, \href{www.i-comse.org}{www.i-comse.org}) was established with funding from an NSF Cybertraining grant, and focuses on serving the needs of the community engaged in fundamental research in computational molecular science and engineering (CMSE) by (a) providing an ecosystem for nurturing and training the next generation of computational molecular scientists and engineers, (b) increasing availability and accessibility of well-curated education material and content, (c) serving as a node for engagement with the relevant stakeholders and promoting fundamental research in CMSE, and (d) broadening participation through recruitment and training of underrepresented groups in CMSE at institutions with low cyberinfrastructure (CI) penetration. 
The institute provides training and educational modules for guided training in advanced computational tools and bolsters the fundamental understanding of the underlying theoretical concepts. Specifically, we offer (1) summer/winter schools for hands-on training on advanced computational techniques and enhancing peer networking for early-stage researchers, and DEI training (2) web-based content to support training at a larger scale, and (3) curriculum and instructional materials for senior undergraduate (UG) and graduate courses to support course development across the community. The planned Summer/winter schools and course materials include modeling approaches spanning different length and time scales (quantum mechanics, atomistic modeling, mesoscale simulations) and advanced data analysis tools, including machine learning. 

 We encourage community members to explore our website for information on past and future schools and materials developed by the institute. If you are an instructor looking for content for your class or course offering, you intend to teach a particular topic, or wish to join for trainer training please email info@i-comse.org