\textbf{Contributed by Andrea Cavalli, CECAM Director and Sara Bonella, CECAM Deputy Director}

Created in 1969, the Centre Européen de Calcul Atomique et Moléculaire (CECAM) is the longest-standing international organization for fundamental and applied research in simulation and modeling. 
Twenty-five member organizations fund it, and its activities are articulated in a network of 17 nodes in Europe and Israel. 
A US-based node will join the network soon. 
CECAM Headquarters is located at EPF-Lausanne. 
Core activities at CECAM are i) scientific workshops and schools on algorithmic and software development and applications; ii) brainstorming and problem-solving events; iii) collaborative research projects for Europe and beyond; iv) the sponsorship of an international visitors’ program. 
A growing portfolio of online and hybrid activities complements in-person events. 
CECAM is also developing teaching material for computational sciences via the \href{https://www.osscar.org/}{Open Software Services for Classrooms and Research (OSSCAR) project} \cite{DU2023}. 
The beta version of Lhumos, a web portal for training and research in computational sciences developed by CECAM in collaboration with the Swiss National Center for Competence and Research MARVEL and the EU-funded Center of Excellence for Computing applications MAX will be online in June 2023.
 
CECAM brings together researchers and students at different stages of their careers, acting as an incubator for exciting science and learning via lively discussion in an informal and productive environment. 
Each year, about 3000 colleagues participate in our activities. 
Explore our website, \href{https://www.cecam.org/}{www.cecam.org}, to discover CECAM’s program and join our mailing lists if you want to keep up with CECAM’s activities. 
Input and new ideas are always welcome. 
Interested scientists, including early career researchers, are welcome to submit proposals for our yearly call for events and contact director@cecam.org to explore collaborations. 
The next call for proposals will be out at the end of April 2023.