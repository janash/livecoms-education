
\textbf{Contributed by Bonnie Hall}

The \href{(https://www.basilbiochem.org/home)}{Biochemistry Authentic Scientific Inquiry Laboratory (BASIL)} is a course-based research experience (CURE) that uses computational techniques and traditional laboratory methods to assign functions to proteins \cite{Roberts2019}. 
The BASIL curriculum aims to get students to transition from thinking like students to thinking like scientists. 
Students analyze proteins with known structures but unknown functions, using modules for computational analyses and/or wet-lab techniques. 
BASIL is designed for undergraduate biochemistry lab courses, but can be adapted to first year (or even high school) settings, as well as upper-level undergraduate or graduate coursework. It is relevant for students in biology, biochemistry, chemistry, or related majors. The curriculum is flexible, allowing module(s) to be used in any order and for only select modules to be utilized.
The BASIL Community provides curricular materials and instructor support for teaching biochemistry laboratory courses using course-based research. Computational and wet lab modules are available on the \href{https://www.basilbiochem.org/basil-modules}{BASIL Modules page}. 
BASIL also offers a variety of professional development activities throughout the year to help instructors learn about and implement our curriculum. This includes a \href{https://join.slack.com/t/onlinebasildiscussion/shared_invite/zt-1skdrd7br-x9MWjHCcNjmV0gkNftNMaw}{Slack channel to connect with the community}, learn more about BASIL, ask questions, and get help with using modules in your course.