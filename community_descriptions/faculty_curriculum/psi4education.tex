\textbf{Contributed by Brandon Magers}

Psi4Education is the education and outreach program of \href{psicode.org}{Psi4}, 
a free, open-source quantum chemistry software package.  Our goal is to provide a suite of free, high-quality, computational chemistry lab activities suitable across undergraduate chemistry courses without requiring a significant overhaul of the existing curriculum. 
Through these activities, Psi4Education aims to increase exposure to computational chemistry and scientific programming for undergraduate chemistry majors.  
The use of computing skills is becoming increasingly important in modern research. 
Most high-impact articles combine computation with experiment, implying that even modern experimental researchers must be at least literate in many of the finer details of computational chemistry. 
Additionally, scientific programming is standard in other disciplines (such as physics) and provides students with another tool available to help them enhance their problem-solving skills. 
A generation ago, including spreadsheets in chemical education enhanced chemical problem-solving, and scientific programming promises to do the same for contemporary students. 
Finally, computational chemistry is a fantastic educational tool to teach students chemistry. 
The ability to visualize changing geometries, the actual motions of atomic vibrations, and the spatial extent and nodal structure of molecular orbitals, among many others, are examples of chemical concepts that can increase student comprehension through a mathematical and graphical presentation followed by manipulation afforded through chemical computation.

The Psi4Education community comprises faculty, graduate students, undergraduate students, and research scientists from across the United States and even the world beyond.  
We encourage joining our community on \href{https://github.com/Psi4Education}{GitHub}.  
Every laboratory exercise and activity is freely available and open-source.  
All activities are web-based through Jupyter Notebooks or WebMO and are locally software-installation-free to lower the barrier to implementation in a classroom setting.  
Concepts on bond breaking, the Hartree-Fock method, atomic radii, polarity, symmetry, and machine learning are only a few currently available activities.  
These exercises are primarily focused on physical chemistry, but concepts related to general, organic, and inorganic chemistry are all available. 
However, the Psi4Edication community is encouraging and welcoming to those who wish to add and augment these exercises with contributions of their own.
