\textbf{Contributed by Sally S. Hunnicutt, Alex Grushow, Marc Muniz, Robert Whitnell}

POGIL-PCL (POGIL for the Physical Chemistry Laboratory) represents a community of over 200 instructors involved in the development and implementation of experiments for this course. POGIL-PCL also comprises a set of experiments developed using POGIL principles that emphasize modeling of both macroscopic and microscopic chemical phenomena, student design and refinement of experimental protocols, and data pooling to assist in uncovering physical chemistry principles. 

The experiments follow a cycle. The experiment begins with a titular question. Students answer four to ten pre-experiment questions, review the experimental protocol, predict an outcome, and make choices about performing the experiment such as concentration, temperature, or time. After carrying out the experiment, the students work in teams to answer Thinking About The Data (TATD) questions. The cycle is repeated two to three times, and in each cycle students develop increasingly sophisticated models of their results. Each of the completed experiments includes an instructor handbook with sample data and typical student responses, and each experiment has undergone numerous rounds of development, implementation at different colleges, reviews, and revisions. A number of the experiments have been published in peer-reviewed journals or books. All are available upon request (visit pogilpcl.org to request access).

Our work also includes two research initiatives: probing student learning outcomes in this upper-level laboratory-based course and uncovering the key features that both grow and sustain the POGIL-PCL community. Both the in-person workshops and their online counterparts provide the experience of doing an experiment including teamwork, experiment design, and data analysis. Participants prefer face-to-face workshops for their opportunity for connection and professional development, but the POGIL-PCL community continues to grow and instructors continue to implement POGIL-PCL experiments when workshops are virtual. The POGIL-PCL community looks to remain active by hosting quarterly online workshops and monthly journal club discussions of selected literature or topics of interest.