%%%%%%%%%%%%%%%%%%%%%%%%%%%%%%%%%%%%%%%%%%%%%%%%%%%%%%%%%%%%
%%% LIVECOMS ARTICLE TEMPLATE FOR BEST PRACTICES GUIDE
%%% ADAPTED FROM ELIFE ARTICLE TEMPLATE (8/10/2017)
%%%%%%%%%%%%%%%%%%%%%%%%%%%%%%%%%%%%%%%%%%%%%%%%%%%%%%%%%%%%
%%% PREAMBLE
\documentclass[9pt,review]{livecoms}
% Use the 'onehalfspacing' option for 1.5 line spacing
% Use the 'doublespacing' option for 2.0 line spacing
% Use the 'lineno' option for adding line numbers.
% Use the 'pubversion' option for adding the citation and publication information to the document footer, when the DOI is assigned and the article is added to a live issue.
% The 'bestpractices' option for indicates that this is a best practices guide.
% Omit the bestpractices option to remove the marking as a LiveCoMS paper.
% Please note that these options may affect formatting.

\usepackage{lipsum} % Required to insert dummy text
\usepackage[version=4]{mhchem}
\usepackage{siunitx}
\DeclareSIUnit\Molar{M}
\usepackage[italic]{mathastext}
\graphicspath{{figures/}}


%%%%%%%%%%%%%%%%%%%%%%%%%%%%%%%%%%%%%%%%%%%%%%%%%%%%%%%%%%%%
%%% IMPORTANT USER CONFIGURATION
%%%%%%%%%%%%%%%%%%%%%%%%%%%%%%%%%%%%%%%%%%%%%%%%%%%%%%%%%%%%

\newcommand{\versionnumber}{1.3}  % you should update the minor version number in preprints and major version number of submissions.
\newcommand{\githubrepository}{\url{https://github.com/myaccount/homegithubrepository}}  %this should be the main github repository for this article

%%%%%%%%%%%%%%%%%%%%%%%%%%%%%%%%%%%%%%%%%%%%%%%%%%%%%%%%%%%%
%%% ARTICLE SETUP
%%%%%%%%%%%%%%%%%%%%%%%%%%%%%%%%%%%%%%%%%%%%%%%%%%%%%%%%%%%%
\title{Learning Communities in Computational Molecular Sciences [Article v\versionnumber]}

\author[1*]{Firstname Middlename Surname}
\author[1,2\authfn{1}\authfn{3}]{Firstname Middlename Familyname}
\author[2\authfn{1}\authfn{4}]{Firstname Initials Surname}
\author[2*]{Firstname Surname}
\affil[1]{Institution 1}
\affil[2]{Institution 2}

\corr{email1@example.com}{FMS}  % Correspondence emails.  FMS and FS are the appropriate authors initials.
\corr{email2@example.com}{FS}

\orcid{Author 1 name}{AAAA-BBBB-CCCC-DDDD}
\orcid{Author 2 name}{EEEE-FFFF-GGGG-HHHH}

\contrib[\authfn{1}]{These authors contributed equally to this work}
\contrib[\authfn{2}]{These authors also contributed equally to this work}

\presentadd[\authfn{3}]{Department, Institute, Country}
\presentadd[\authfn{4}]{Department, Institute, Country}

\blurb{This LiveCoMS document is maintained online on GitHub at \githubrepository; to provide feedback, suggestions, or help improve it, please visit the GitHub repository and participate via the issue tracker.}

%%%%%%%%%%%%%%%%%%%%%%%%%%%%%%%%%%%%%%%%%%%%%%%%%%%%%%%%%%%%
%%% PUBLICATION INFORMATION
%%% Fill out these parameters when available
%%% These are used when the "pubversion" option is invoked
%%%%%%%%%%%%%%%%%%%%%%%%%%%%%%%%%%%%%%%%%%%%%%%%%%%%%%%%%%%%
\pubDOI{10.XXXX/YYYYYYY}
\pubvolume{<volume>}
\pubissue{<issue>}
\pubyear{<year>}
\articlenum{<number>}
\datereceived{Day Month Year}
\dateaccepted{Day Month Year}

%%%%%%%%%%%%%%%%%%%%%%%%%%%%%%%%%%%%%%%%%%%%%%%%%%%%%%%%%%%%
%%% ARTICLE START
%%%%%%%%%%%%%%%%%%%%%%%%%%%%%%%%%%%%%%%%%%%%%%%%%%%%%%%%%%%%

\begin{document}

\begin{frontmatter}
\maketitle

\begin{abstract}
This particular document provides a skeleton illustrating key sections for a Perpetual Review document.
Please see the sample \texttt{sample-document.tex} in \url{github.com/livecomsjournal/article_templates/templates} for additional information on and examples of using the LiveCoMS LaTeX class.
Here we also assume familiarity with LaTeX and knowledge of how to include figures, tables, etc.; if you want examples, see the sample just referenced.

In your work, in this particular slot, please provide an abstract of no more than 250 words.
Your abstract should explain the main contributions of your article, and should not contain any material that is not included in the main text.
Please note that your abstract, plus the authorship material following it, must not extend beyond the title page or modifications to the LaTeX class will likely be needed.
\end{abstract}

\end{frontmatter}


\section{Introduction}

Here you would explain what area you are reviewing, and motivate your review.

In this particular template, we have removed most of the usage examples which occur in \texttt{sample-document.tex} to provide a minimal template you can modify.

Keep in mind, as you prepare your manuscript, that you should plan for a representative image  which will be used to highlight your article on the journal website and publications. Usually, this would be one of your figures, but it must also be uploaded separately upon article submission. We give specific guidelines for this image on the journal website in the section on article submission (see \url{https://livecomsjournal.github.io/authors/policies/index.html#article-submission}).

Additionally, for well-formatted manuscripts, we recommend that you let LaTeX handle figure/table placement for you as much as possible, so please avoid specifying strenuous float instructions like `[h!]` and `[H]` as much as possible.

\section{Review organization}

In general the organization of your review is up to you.
LiveCoMS reviews are similar to traditional reviews except that they are updatable.

\subsection{Perpetual review}

You will probably want to devote at least a small amount of space explaining to your readers what a Perpetual Review is and highlighting how they can get involved via your git issue tracker or by communicating with you directly.
In contrast to a traditional review, here it \emph{will} help you if the community points out other, related work that perhaps should be cited, since it reduces the need for you to dig deeply into the literature to keep your work up to date.

Tutorials should endeavor to cover the specific task at hand, and also highlight how the steps might need to be modified (or additional care might need to be taken at particular points) to handle more general cases.

The scope of the tutorial, as well as the expected proficiencies / outcomes for researchers who complete the tutorial, should be clearly defined.
This will often happen in a specific section or subsection in the article itself.

\section{Prerequisites}

%In keeping with other types of articles at LiveCoMS, it may be useful (at your discretion) to note any prerequisites your article assumes and who it is targeted at.

This article is aimed at improving the centralization of resources for computational molecular sciences. It is designed for both students and educators, and has a particular focus on resources that create learning communities. 

Here, we present the minimum eligibility criteria for educational communities and organizations that offer resources and learning opportunities on computational molecular science:
\begin{itemize}
\item The community should be a part of an academic organization or multiple academic institutions;
\item The community should provide resources explaining concepts beyond the software/tools used for calculations;
\item The community should use open licenses for educational resources (e.g. CC-BY). 
\end{itemize}

Considering the implicit diversity of educational communities we expect that one size does not fit all, but all educational communities and organizations listed here satisfy at least one criterion in each of the following four domains: accessibility, activity, growth and pedagogy. %Should we include the link to public survey here, for people to self-determine if they count? I have added it below -A

To self-determine if you fulfill the criteria, we have provided a public survey here:

\url{https://molssi.typeform.com/to/vSSii68D?typeform-source=t.co}

\subsection{Accessibility}
The community should provide accessible resources and a responsive attitude towards trainees with diverse needs. At the bare minimum, this means that all softwares/tools that are being used in learning activities are open-source or free for academic use and/or most trainees have easy access to the computational tools that are being taught. 
\subsection{Activity}
The community should actively organize activities or offer resources that can be used for activities by others. These activities can be online resources or virtual/in-person events. In the case of the former, the community should provide online resources that are accessible to learners and/or educators. It is acceptable if the instructor resources have limited distribution, i.e. one must register as an instructor to the community to gain access. In the case of the latter, the community should organize regular events that are open to students, researchers, and/or educators.
\subsection{Growth} 
The community must be capable of growing and being sustainable. This can be in the form of training new members, educators and/or contributors. Alternatively, the community could offer a way to get involved by contributing and editing resources, or attending events, and should provide this information publicly.
\subsection{Pedagogy}
The community should pay attention to the teaching style used in their resources. The resources should include clearly-defined learning objectives and/or offer support for learners past specific learning activities. The support could be via a mailing list, online forum or an issue tracker.
\subsection{Updates}
As this is a Perpetual Review, the contents will be updated annually with new or updated contributions. To be included in the article in the future, or to update your contribution, please use the following Github pull request template:

%Add the link to the Github template here!

A review author will contact you to handle your contribution given that you fulfill the above requirements for a learning community. 

\section{Education Communities}

The education communities fall into three broad categories:
\begin{itemize}
    \item Curriculum Development and Faculty Communities
    \item Tutorials, Schools, and Online Resources
    \item Research Centers and Institutes
\end{itemize}

\subsection{Curriculum Development and Faculty Communities}

\subsubsection{Biochemistry Authentic Laboratory Scientific Inquiry (BASIL)}

\textbf{Contributed by Bonnie Hall}

The \href{(https://www.basilbiochem.org/home)}{Biochemistry Authentic Scientific Inquiry Laboratory (BASIL)} is a course-based research experience (CURE) that uses computational techniques and traditional laboratory methods to assign functions to proteins \cite{Roberts2019}. 
The BASIL curriculum aims to get students to transition from thinking like students to thinking like scientists. 
Students analyze proteins with known structures but unknown functions, using modules for computational analyses and/or wet-lab techniques. 
BASIL is designed for undergraduate biochemistry lab courses, but can be adapted to first year (or even high school) settings, as well as upper-level undergraduate or graduate coursework. It is relevant for students in biology, biochemistry, chemistry, or related majors. The curriculum is flexible, allowing module(s) to be used in any order and for only select modules to be utilized.
The BASIL Community provides curricular materials and instructor support for teaching biochemistry laboratory courses using course-based research. Computational and wet lab modules are available on the \href{https://www.basilbiochem.org/basil-modules}{BASIL Modules page}. 
BASIL also offers a variety of professional development activities throughout the year to help instructors learn about and implement our curriculum. This includes a \href{https://join.slack.com/t/onlinebasildiscussion/shared_invite/zt-1skdrd7br-x9MWjHCcNjmV0gkNftNMaw}{Slack channel to connect with the community}, learn more about BASIL, ask questions, and get help with using modules in your course.

\subsubsection{Enhancing Science Courses by Integrating Python (ESCIP)}
\textbf{Contributed by Davit Potoyan}

Enhancing Science Courses by Integrating Python (ESCIP) is a resource and a community of instructors who work together to lower the barriers for integrating computational science concepts and skills into STEM courses. 
We aim to help bridge the increasing gap between computational skills in demand and skills taught at universities. 
The \href{https://escip.github.io/}{ESCIP website} contains a centralized resource hosted in Github as a Jupyter-book. 
The Jupyter book binds together self-contained single-lecture Jupyter notebooks created and edited, and used by different instructors to teach STEM courses. 
All notebooks can be run in the cloud via binder or GoogleColab, which allows instructors new to Python to focus on playing to the strength of their subject by emphasizing deep links between basic science and math concepts and data science techniques. 
Furthermore, the resources of ESCIP provide a natural context and ready-to-use tools for integrating computation, simulation, and data science that feels familiar and accessible to students and instructors.


\subsubsection{MolecVUE}
\textbf{Contributed by Kevin M. Range}

The MoleCVUE consortium is an association of faculty at Primarily Undergraduate Institutions (PUIs) who share the common purpose expressed by our name: Molecular Computation and Visualization in Undergraduate Education. 
Participants at recent events have come from a variety of academic institutions including in faculty and students alike. 
Our areas of interest include not only chemistry, but also any area where the ruling paradigm is “molecules.” 
Our objectives are to enhance our own undergraduate instruction by developing and employing modern computational and visualization tools and to provide the methods, techniques, and resources we develop freely to other science instructors.

The MoleCVUE consortium strives to provide a community for expressing one’s own ideas and receiving constructive criticism from peer educators. Our \href{https://sites.google.com/view/molecvue/events?authuser=0}{website} hosts \href{https://sites.google.com/view/molecvue/resources?authuser=0}{curated resources} for incorporating computational chemistry into your curriculum. We host hybrid meetings and symposia with a workshop format to facilitate collaborations between participants and to hone existing educational methods and tools. We actively develop computational chemistry activities for the modern chemistry learning space.

\subsubsection{Partnership for Integration of Computation into Undergraduate Physics (PICUP)}
\textbf{Contributed by Danny Caballero}

The Partnership for Integration of Computation into Undergraduate Physics (PICUP) is dedicated to the principle that computation should be an integral part of the education of every undergraduate physics student. Their mission is to foster a vibrant community of educators, provide a forum for open discussion, build a repository of educational resources, and offer strategies to support the development and enhancement of undergraduate physics education by integrating computation across the curriculum.

PICUP's approach is to increase the use of computation in individual physics courses. They acknowledge that computation is indispensable for physicists in the 21st century, and it must be an integral part of physics education. In pursuit of their goals, PICUP develops computational packages that are intricately connected with the subject matter of popular physics texts and removes the barriers to the adoption and customization of educational materials. PICUP's partnership consists of a diverse team of physics instructors who share an interest and a commitment to the integration of computation into undergraduate courses, representing the wide range of institutional types and departmental environments across the country. PICUP provides direct support through professional development opportunities, including workshops (both in-person and virtual) and longer departmental and regional retreats. 

If you're interested in learning more about PICUP, please visit \href{https://gopicup.org}{gopicup.org}.

\subsubsection{Process-Oriented Guided Learning Inquiry for the Physial Chemistry Laboratory (POGIL-PCL)}
\textbf{Contributed by Sally S. Hunnicutt, Alex Grushow, Marc Muniz, Robert Whitnell}

POGIL-PCL (POGIL for the Physical Chemistry Laboratory) represents a community of over 200 instructors involved in the development and implementation of experiments for this course. POGIL-PCL also comprises a set of experiments developed using POGIL principles that emphasize modeling of both macroscopic and microscopic chemical phenomena, student design and refinement of experimental protocols, and data pooling to assist in uncovering physical chemistry principles. 

The experiments follow a cycle. The experiment begins with a titular question. Students answer four to ten pre-experiment questions, review the experimental protocol, predict an outcome, and make choices about performing the experiment such as concentration, temperature, or time. After carrying out the experiment, the students work in teams to answer Thinking About The Data (TATD) questions. The cycle is repeated two to three times, and in each cycle students develop increasingly sophisticated models of their results. Each of the completed experiments includes an instructor handbook with sample data and typical student responses, and each experiment has undergone numerous rounds of development, implementation at different colleges, reviews, and revisions. A number of the experiments have been published in peer-reviewed journals or books. All are available upon request (visit pogilpcl.org to request access).

Our work also includes two research initiatives: probing student learning outcomes in this upper-level laboratory-based course and uncovering the key features that both grow and sustain the POGIL-PCL community. Both the in-person workshops and their online counterparts provide the experience of doing an experiment including teamwork, experiment design, and data analysis. Participants prefer face-to-face workshops for their opportunity for connection and professional development, but the POGIL-PCL community continues to grow and instructors continue to implement POGIL-PCL experiments when workshops are virtual. The POGIL-PCL community looks to remain active by hosting quarterly online workshops and monthly journal club discussions of selected literature or topics of interest.

\subsubsection{Psi4Education}
\textbf{Contributed by Brandon Magers}

Psi4Education is the education and outreach program of \href{psicode.org}{Psi4}, 
a free, open-source quantum chemistry software package.  Our goal is to provide a suite of free, high-quality, computational chemistry lab activities suitable across undergraduate chemistry courses without requiring a significant overhaul of the existing curriculum. 
Through these activities, Psi4Education aims to increase exposure to computational chemistry and scientific programming for undergraduate chemistry majors.  
The use of computing skills is becoming increasingly important in modern research. 
Most high-impact articles combine computation with experiment, implying that even modern experimental researchers must be at least literate in many of the finer details of computational chemistry. 
Additionally, scientific programming is standard in other disciplines (such as physics) and provides students with another tool available to help them enhance their problem-solving skills. 
A generation ago, including spreadsheets in chemical education enhanced chemical problem-solving, and scientific programming promises to do the same for contemporary students. 
Finally, computational chemistry is a fantastic educational tool to teach students chemistry. 
The ability to visualize changing geometries, the actual motions of atomic vibrations, and the spatial extent and nodal structure of molecular orbitals, among many others, are examples of chemical concepts that can increase student comprehension through a mathematical and graphical presentation followed by manipulation afforded through chemical computation.

The Psi4Education community comprises faculty, graduate students, undergraduate students, and research scientists from across the United States and even the world beyond.  
We encourage joining our community on \href{https://github.com/Psi4Education}{GitHub}.  
Every laboratory exercise and activity is freely available and open-source.  
All activities are web-based through Jupyter Notebooks or WebMO and are locally software-installation-free to lower the barrier to implementation in a classroom setting.  
Concepts on bond breaking, the Hartree-Fock method, atomic radii, polarity, symmetry, and machine learning are only a few currently available activities.  
These exercises are primarily focused on physical chemistry, but concepts related to general, organic, and inorganic chemistry are all available. 
However, the Psi4Edication community is encouraging and welcoming to those who wish to add and augment these exercises with contributions of their own.


\subsection{Tutorials, Schools, and Online Resources}

\subsubsection{The eChem Project}

\textbf{Contributed by Patrick Norman}

The \href{https://doi.org/10.30746/978-91-988114-0-7}{eChem project} offers a platform for interactive studies in computational chemistry based on Jupyter notebooks and the use of modern Python-driven electronic structure softwares such as VeloxChem  and Gator.  
The notebooks are compiled into a Jupyter Book that targets a wide audience ranging from undergraduate students to experienced researchers.
The central educational idea of eChem is to leverage the Python-layer exposure of data structures in VeloxChem and other software modules to enable the study of quantum chemistry in a similar manner that NumPy (or Matlab) can be used in studies of linear algebra. Presentations of theory and equations are intertwined with notebook cells showing the form these equations take in Python code implementations and fundamental concepts such as two-particle densities and electron correlation can be made visual with use of Matplotlib illustrations. 

While the eChem book is well suited for self-directed learning, the material can also be used to put together workshops led by experienced instructors or as components in standard courses in the chemistry curriculum.  Students are recommended to use conda to install the software on their own laptops running Windows, MacOS, or Linux. 
The eChem book is published under the Creative Commons Attribution-ShareAlike license with the source code available on GitHub.  Community contributions to the project are welcome both as comments and corrections as well as new book sections. The former category is handled through GitHub whereas the latter is initiated by e-mail correspondence to one of the authors.

\subsubsection{Helsinki Winter School in Theoretical Chemistry}

\textbf{Contributed by Susi Lehtola}

The \href{http://www.chem.helsinki.fi/ws.html}{Helsinki Winter School in
Theoretical Chemistry} has been organized annually
since 1985 at the \href{http://www.chem.helsinki.fi/}{Department of Chemistry}
of the \href{www.helsinki.fi/}{University of Helsinki}, Finland, with the
exception of 2020 and 2021 due to the coronavirus pandemic. The theme of the
Winter School changes from year to year, in order to cover various fields of
theoretical chemistry as well as possible. For example, the last four schools
covered quantum computers for chemistry (2022), inorganic chemistry (2019),
machine learning in theoretical chemistry (2018), and current trends in
electronic structure methods (2017). The school aims to invite the leading 
experts in the field, guaranteeing lectures of high scientific quality. The
lecture materials are typically only made available to the workshop attendees,
so that lecturers feel more free to present unpublished results.

The winter school is typically organized in late October to early December as a
four-day event, and usually has around 80 participants, around half of whom come
from abroad. The school is targeted for PhD students and young researchers, but
they usually also include a few more established researchers and professors; the
presence of the more senior personnel often leads to lively scientific
discussions during the lectures. Participation in the winter school is free;
however, attendees need to pay their own travel and accomodation. The school
also features a poster session, which welcomes posters representing any area of
theoretical chemistry.

\subsubsection{Making it Rain}

\textbf{Contributed by Pablo Arantes}

Molecular dynamics (MD) simulations are widely used for describing biomolecular motion, employing a classic physics approximation to depict atomic trajectories \cite{Liu2017,Hollingsworth2018}. However, due to their high computational demands, there is an increasing effort to enable large-scale runs of MD simulations \cite{Hollingsworth2018}. Cloud computing has emerged as a potential solution due to its scalability, reliability, and cost-effectiveness \cite{Ebejer2013}. Nevertheless, utilizing cloud computing for MD simulations requires technical expertise and the ability to perform tedious tasks, such as configuring remote nodes and installing software. To address this challenge, we developed a user-friendly front-end for running MD simulations \cite{Arantes2021} using the OpenMM toolkit \cite{Eastman2017} on the Google Colab framework \cite{colab,Bisong2019} (\url{https://pablo-arantes.github.io/making-it-rain/}). Our approach involves shareable, ready-to-use, and customizable Jupyter notebooks that guide the end-user through the process of running their calculations on the Google Colab platform. This approach makes it more accessible for low-income research groups to perform MD simulations at the microsecond timescale, and also facilitates teaching and learning of molecular simulations. These protocols can be customized for specific use cases, providing students with a hands-on approach to learning MD simulations.

We believe that our work has a significant impact in the scientific community by enabling more researchers to conduct MD simulations without the high cost of traditional computing resources. This approach is also valuable for educators who wish to provide students with a hands-on experience in MD simulations. In summary, our approach offers an affordable and accessible solution for performing MD simulations using cloud computing resources. We hope that this strategy will empower researchers and educators alike, ultimately leading to new insights and discoveries in the field of biomolecular dynamics.

All notebooks presented here are freely and publicly available at  \url{https://pablo-arantes.github.io/making-it-rain/}.

\subsubsection{PLUMED}

\textbf{Contributed by }

The PLUMED consortium\cite{plumed2019promoting} is a community born around the PLUMED open source software (\href{https://www.plumed.org}{www.plumed.org})\cite{tribello2014plumed}, a library to perform enhanced sampling and free energy calculations, and to analyze the data produced by molecular dynamics (MD) simulations. PLUMED works together with several molecular simulation packages, including but not limited to GROMACS, NAMD, OpenMM, Amber, CP2K, and Quantum ESPRESSO. The PLUMED consortium  is an open community that includes current and past PLUMED developers, contributors, and all those researchers whose work builds in part on PLUMED and at the same time drives the development and dissemination of the software. The mission of the consortium is to transform the way researchers communicate the protocols that are used in their MD simulations, in order to maximize the impact of new research and promote the highest possible standards of scientific reproducibility. 

To achieve these goals, the PLUMED consortium coordinates two main initiatives. The first one is a public repository named PLUMED-NEST (\href{https://www.plumed-nest.org}{www.plumed-nest.org}) where contributors can share examples (including input and data files) to promote reproducibility and transparency in computational molecular science. The second initiative is a series of tutorials, called PLUMED Masterclasses, that teach molecular simulation techniques ranging from basic algorithms to advanced enhanced sampling methods. Recordings of lectures about these techniques can be found on the Youtube channel @plumedorg1402, and all the teaching material, including exercises, solutions, and slides, are available at www.plumed.org/masterclass. Further training and interactions between members of the PLUMED community revolve around the PLUMED forum (plumed-users@googlegroups.com) and the periodic, in-person schools and user meetings, which give the opportunity to new and experienced users to meet and discuss specific use cases.

\subsubsection{TeachOpenCADD}
\textbf{Contributed by Dominique Sydow and Andrea Volkamer}

TeachOpenCADD is a free online platform launched in 2019 that presents common topics from computer-aided drug design (CADD) and how to solve them programmatically in Python with open-source data and packages \cite{TeachOpenCADD2019, TeachOpenCADD2022, TeachOpenCADDKinaseEdition}. The current lessons range from cheminformatics over structural bioinformatics to molecular modeling. The material also exemplifies how to query domain-relevant databases such as ChEMBL \cite{Mendez_2018_ChEMBL}, PubChem \cite{Kim_2021_PubChem}, and the PDB \cite{Burley_2020_PDB}. 
Each lesson is presented in the form of an interactive Jupyter notebook \cite{Kluyver_2016_Jupyter}, which combines detailed theoretical background and executable code with a discussion and a quiz. The material is suitable for students and teachers from different disciplines and programming levels ranging from beginners to advanced users. It can be applied to self-study and classroom teaching but also serves as a basis for debarking on research questions \cite{TeachOpenCADDTeaching}. 

The TeachOpenCADD material is available in many forms: (i) The TeachOpenCADD website \cite{TOC_Website} allows for instant browsing of the full content in read-only mode. (ii) Users can execute and modify the material either online via the Binder \cite{binder} support or locally via the TeachOpenCADD conda package \cite{conda_forge_community_2015_4774216, conda_forge_toc} as described in detail in the TeachOpenCADD documentation \cite{TOC_Documentation}. (iii) If users are looking for a coding-free solution, the cheminformatics-focused topics are available as GUI-based KNIME workflows \cite{TeachOpenCADDKNIME2019}. The TeachOpenCADD platform is actively maintained and continuously extended with new content (e.g. the kinase edition in 2022 \cite{TeachOpenCADDKinaseEdition} and a deep learning edition under current development). It is open to contributions, bug reports, and questions from the community via its GitHub repository \cite{TOC_Website}.

\subsubsection{Virtual Winter School on Computational Chemistry}

\textbf{Contributed by Rebecca Ingle}

The \href{https://winterschool.cc}{Virtual Winter School on Computational Chemistry} (VWSCC) is an online conference series and community. Founded in 2015, the core aim of VWSCC is to improve the accessibility of scientific research to the scientific community all over the world by removing many of the traditional barriers to conference attendance - geography, income, career stage and many others. As well as offering access to talks combining pedagogical content and cutting-edge research developments, the VWSCC has been creating and maintaining an archive of education resources and running hands-on workshops for a variety of quantum chemical codes.  Participation is free and includes an online poster session. In this paper\cite{ROOS2020112975}, we discuss the reach of the VWSCC, some of the resources we offer and how to get involved in the community, an evaluation of how to run successful and accessible online events, and a summary of some of the challenges of working within this format.



\subsection{Research Centers and Institutes}

\subsubsection{Centre Européen de Calcul Atomique et Moléculaire (CECAM)}

\textbf{Contributed by Andrea Cavalli, CECAM Director and Sara Bonella, CECAM Deputy Director}

Created in 1969, the Centre Européen de Calcul Atomique et Moléculaire (CECAM) is the longest-standing international organization for fundamental and applied research in simulation and modeling. 
Twenty-five member organizations fund it, and its activities are articulated in a network of 17 nodes in Europe and Israel. 
A US-based node will join the network soon. 
CECAM Headquarters is located at EPF-Lausanne. 
Core activities at CECAM are i) scientific workshops and schools on algorithmic and software development and applications; ii) brainstorming and problem-solving events; iii) collaborative research projects for Europe and beyond; iv) the sponsorship of an international visitors’ program. 
A growing portfolio of online and hybrid activities complements in-person events. 
CECAM is also developing teaching material for computational sciences via the \href{https://www.osscar.org/}{Open Software Services for Classrooms and Research (OSSCAR) project} \cite{DU2023}. 
The beta version of Lhumos, a web portal for training and research in computational sciences developed by CECAM in collaboration with the Swiss National Center for Competence and Research MARVEL and the EU-funded Center of Excellence for Computing applications MAX will be online in June 2023.
 
CECAM brings together researchers and students at different stages of their careers, acting as an incubator for exciting science and learning via lively discussion in an informal and productive environment. 
Each year, about 3000 colleagues participate in our activities. 
Explore our website, \href{https://www.cecam.org/}{www.cecam.org}, to discover CECAM’s program and join our mailing lists if you want to keep up with CECAM’s activities. 
Input and new ideas are always welcome. 
Interested scientists, including early career researchers, are welcome to submit proposals for our yearly call for events and contact director@cecam.org to explore collaborations. 
The next call for proposals will be out at the end of April 2023.


\subsubsection{The Institute of Computational Molecular Science Education}

\textbf{Contributed by Neeraj Rai}

The Institute of Computational Molecular Science Education (i-CoMSE, \href{www.i-comse.org}{www.i-comse.org}) was established with funding from an NSF Cybertraining grant, and focuses on serving the needs of the community engaged in fundamental research in computational molecular science and engineering (CMSE) by (a) providing an ecosystem for nurturing and training the next generation of computational molecular scientists and engineers, (b) increasing availability and accessibility of well-curated education material and content, (c) serving as a node for engagement with the relevant stakeholders and promoting fundamental research in CMSE, and (d) broadening participation through recruitment and training of underrepresented groups in CMSE at institutions with low cyberinfrastructure (CI) penetration. 
The institute provides training and educational modules for guided training in advanced computational tools and bolsters the fundamental understanding of the underlying theoretical concepts. Specifically, we offer (1) summer/winter schools for hands-on training on advanced computational techniques and enhancing peer networking for early-stage researchers, and DEI training (2) web-based content to support training at a larger scale, and (3) curriculum and instructional materials for senior undergraduate (UG) and graduate courses to support course development across the community. The planned Summer/winter schools and course materials include modeling approaches spanning different length and time scales (quantum mechanics, atomistic modeling, mesoscale simulations) and advanced data analysis tools, including machine learning. 

 We encourage community members to explore our website for information on past and future schools and materials developed by the institute. If you are an instructor looking for content for your class or course offering, you intend to teach a particular topic, or wish to join for trainer training please email info@i-comse.org

\subsubsection{The Molecular Sciences Software Institute}

\textbf{Contributed by Jessica A. Nash}

 The Molecular Sciences Software Institute (MolSSI) serves and enhances software development efforts in the computational molecular sciences (CMS). MolSSI Education is the institute's division that trains students and researchers in computer programming and software development practices. Our three main focus areas are (1) Programming and Software Development; (2) High-performance computing (HPC); and (3) Faculty and curriculum development \cite{NashCISE}. 

MolSSI Education’s educational material and workshops target a range of CMS students and researchers including undergraduates, graduate students, faculty, and professionals in CMS. 
Our online lessons are conveniently accessible through user-friendly Jupyter notebooks and web pages, which provide step-by-step examples of lesson content. 
Tutorials and workshops are hands-on, focusing on giving the student practical experience and skills. 
Online materials include our Python Data and Scripting for Computational Molecular Science, Best Practices in Software Development, Fundamentals of Parallel Programming, and many others (see \href{https://education.molssi.org/resources}{https://education.molssi.org/resources} for a complete list). 

We strive to collaborate with other institutes, communities, and resources and regularly hold workshops with academic research groups and universities. The MolSSI has also hosted academic faculty on sabbatical, resulting in the development of specialized learning materials for CMS researchers \cite{Craig2022}.  

All MolSSI Education lessons are developed with open licensing and are hosted on GitHub through the \href{https://github.com/molssi-education}{MolSSI Education GitHub Organization}. We welcome external collaboration and contributions through GitHub Pull Requests. If you would like to discuss workshop opportunities with MolSSI, you can contact the MolSSI Education Lead, Dr. Jessica Nash (janash@vt.edu), or MolSSI's Co-Director for Education, Training, and Faculty Development, Prof. Ashley McDonald (armcdona@calpoly.edu)

 \subsubsection{Partnership for Advanced Computing in Europe (PRACE)}

\textbf{Contributed by Marjolein Oorsprong}

Over several years, PRACE (\href{https://prace-ri.eu/}{https://prace-ri.eu/}) has developed an extensive Training Portal (\href{https://training.prace-ri.eu/}{https://training.prace-ri.eu/}) where visitors can explore upcoming training events and find a range of other HPC-related training resources.
The events and resources are offered at different levels of proficiency in the use of HPC, but can be of relevance to any scientific field or industrial domain that relies on supercomputers.
During the COVID-19 pandemic, all PRACE-supported training events were reformatted for online-only participation. 
PRACE remains committed to delivering most courses online but a small number of courses will allow hybrid (in-person and online) participation since September 2021.
PRACE training courses are open to participants affiliated with institutions from the Member States (MS) of the European Union (EU) and \href{https://ec.europa.eu/info/research-and-innovation/statistics/framework-programme-facts-and-figures/horizon-2020-country-profiles_en}{Associated/Other Countries} to the Horizon 2020 programme.
PRACE also supports the International HPC Summer School (\href{https://www.ihpcss.org/}{https://www.ihpcss.org/}).
Looking towards the future, PRACE is currently working with the EuroHPC Joint Undertaking 
(\href{https://eurohpc-ju.europa.eu/index_en}{https://eurohpc-ju.europa.eu/index\_en}).
on a professional HPC traineeship programme, which is planned to start in 2024.

\section{Conclusions}

It's appropriate to wrap up with discussion and conclusions.







\section*{Author Contributions}
%%%%%%%%%%%%%%%%
% This section mustt describe the actual contributions of
% author. Since this is an electronic-only journal, there is
% no length limit when you describe the authors' contributions,
% so we recommend describing what they actually did rather than
% simply categorizing them in a small number of
% predefined roles as might be done in other journals.
%
% See the policies ``Policies on Authorship'' section of https://livecoms.github.io
% for more information on deciding on authorship and author order.
%%%%%%%%%%%%%%%%

(Explain the contributions of the different authors here)

% We suggest you preserve this comment:
For a more detailed description of author contributions,
see the GitHub issue tracking and changelog at \githubrepository.

\section*{Other Contributions}
%%%%%%%%%%%%%%%
% You should include all people who have filed issues that were
% accepted into the paper, or that upon discussion altered what was in the paper.
% Multiple significant contributions might mean that the contributor
% should be moved to authorship at the discretion of the a
%
% See the policies ``Policies on Authorship'' section of https://livecoms.github.io for
% more information on deciding on authorship and author order.
%%%%%%%%%%%%%%%

(Explain the contributions of any non-author contributors here)
% We suggest you preserve this comment:
For a more detailed description of contributions from the community and others, see the GitHub issue tracking and changelog at \githubrepository.

\section*{Potentially Conflicting Interests}
%%%%%%%
%Declare any potentially competing interests, financial or otherwise
%%%%%%%

Declare any potentially conflicting interests here, whether or not they pose an actual conflict in your view.

\section*{Funding Information}
%%%%%%%
% Authors should acknowledge funding sources here. Reference specific grants.
%%%%%%%
FMS acknowledges the support of NSF grant CHE-1111111.

\section*{Author Information}
\makeorcid
Alice R. Walker: 0000-0002-8617-3425

\bibliography{references}

%%%%%%%%%%%%%%%%%%%%%%%%%%%%%%%%%%%%%%%%%%%%%%%%%%%%%%%%%%%%
%%% APPENDICES
%%%%%%%%%%%%%%%%%%%%%%%%%%%%%%%%%%%%%%%%%%%%%%%%%%%%%%%%%%%%

%\appendix


\end{document}
